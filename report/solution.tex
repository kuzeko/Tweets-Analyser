\section{Solution}
\label{sec:solution}


We describe here the PACT program diagram we implemented to produce the statistics on tweets that we need. 
Recall that a PACT program is a generalization of the Map-Reduce paradigm, in which sequence of second order functions are issued in parallel or in sequence and combined to execute complex tasks. 
The programming paradigm defines five second order functions: map, reduce, match, cross and co-group.
It allows the user to specify any kind of combination between them, in any order.
We propose a flow that uses all the operator except the cross.
For ease of explanation we describe the PACT program\footnote{\url{https://github.com/kuzeko/Tweets-Analyser/}} as composed in two different blocks: data cleaning (Section~\ref{sec:cleaning}) and the computation of the statistics (Section~\ref{sec:statistics}).

\subsection{Data cleaning}
\label{sec:cleaning}
In the data cleaning part we take as input a comma separated file having the format $\langle tweet\_id$,$user\_id$,$tweet \rangle$. 
Table~\ref{tbl:tweets} shows an excerpt of the tweet tuples. 

\begin{table*}[!t]
\begin{tabular}{|l|l|l|}
\hline
tweet id & user id & tweet\\
\hline
292375792485298176&858488612& \#KidCudi - \#EraseMe - The Whizz Bells : http://t.co/Lp1zABOV via @youtube\\
292375792481099777&486970282& RT @Kirra\_\_: Today is a day where I need to crawl into my bed and sleep the day away.\\
292375792481099776&336390437& There are poor people, money is the only thing they got.\\
292375792476909568&74570186& I can't do anything without listening to music while I do it.\\
\hline
\end{tabular}
\caption{A sample of the tweet dataset}
\label{tbl:tweets}
\end{table*}

Note that it is not obvious how to find interesting information from arbitrarily short-texts.
Therefore, we propose a first flow that tries to remove useless or uninformative tweets by filtering-out the ones that do not contain a minimum number of english words. 
Figure~\ref{fig:cleaning} depicts the sequence of operations we designed to clean the data in the preliminary phase. 

Once we loaded the tweets into tuples we clean them, from hashtags, user-mentions and URLs. 
Tweets are then used in two separate flows: (1) we split them into words in order to count the english words and (2) we perform a sentiment analysis over the text, in order to evaluate the positive and negative polarity expressed by them, as described in Section~\ref{sec:introduction}.
Note that SentiStrength library is well suited to analyze informal text, so we apply this evaluation on the whole text, before stemming and further cleaning. 

In order to restrict the search space to those tweets we consider relevant, we import a dictionary of english words and we count english words in each tweet. 
If a tweet has a percentage of english words greater than a threshold $\sigma$ we keep it, otherwise we drop it. 

From the pruned tweets we extract the users, and we assign to the cleaned text the polarities found in the previous step. 


\subsection{Compute statistics}
\label{sec:statistics}

After having cleaned the raw data and after evaluating text polarities, we compute the statistics using the tuples we have kept in the aforementioned steps. 
First, we load and match the tuples with the tweet timestamp we get from the database.
For each timestamp, we  keep the date and time up to the hour, this means that any further analysis is condensed in a time window of 1 hour.
Second, we match the hashtags with the polarities in order to understand positive and negative trends of the topics. 
As Table~\ref{tbl:tweets} shows in a tweet we can find more than one hashtag, but we wish to analyze each hashtag separately.
This step is performed by a match operation with the sentiment polarities followed by a sequence of reduce operations to compute different statistics.

The first information we want to extract is about the evolution of the popularity of an hashtag.
To obtain this information we count hour by hour how many tweets contain each hashtag, and similarly we also keep track of how many different users mention the same hashtag.
%In the same way we are keeping track, hour by hour, of how popularity changes for each hashtag.
Similarly, we collect the hourly popularity of each hashtag.
We also record for each hashtag the moment in which it reaches his peak in popularity alongside the date of first and last appearance, marking in this way the lifespan of the hashtag.
Moreover, we use sentiment analysis results to compute three aggregates, namely the average positive sentiment, the average negative sentiment and the emotional divergence present in each of them.

These statistics can give us some insights on opinions about the topic entailed by the hashtag.
In particular, in \cite{DBLP:conf:icwsm:PfitznerGS12}  is demonstrated how the overall sentiment (polarity) in one tweet does not influence its probability to be re-shared.
For this reason, they propose emotional divergence as a novel measure to identify which tweets have a better chance of being retweeted.
Aggregating this information for each hashtag we can gather additional informations in support of one hashtag being relevant and interesting.



