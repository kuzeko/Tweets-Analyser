\section{Data}
\label{sec:data}

We downloaded 33 million of English tweets from the tweet-stream using the streaming APIs\footnote{\url{https://dev.twitter.com/docs/streaming-apis}} in a period of approximately two weeks from December 18, 2012 to January 18, 2013~\footnote{We didn't downloaded any data during the week between the 25-12-2012 and the 02-01-2013}. 
A summary of the main characteristic of the dataset is represented in table~\ref{tbl:dataset}.
The number of users is 16 million, with an average of 2 tweets per user. 
The number of hashtags is one order of magnitude less than the number of tweets meaning that users often talk about same topics or do not mark their tweets with hashtags.
From this dataset we generated three datasets \textsc{100k-Tweets}, \textsc{1M-Tweets}, \textsc{10M-Tweets} with 100 thousands,1 million and 10 million tweets respectively, in order to test performance with various data size. 
We also downloaded a dictionary of 213k English words to filter the meaningful tweets from those irrelevant or containing only symbols. 
To perform the sentiment analysis (i.e., to assign a positive or negative polarity to each tweet) we used the SentiStrength library registering both positive and negative polarity carried in each tweet. 

\begin{table}[htb]
\centering 
\begin{tabular}{|l|r|}
\hline		
Period			& 2012-12-18/2013-01-18\\
\hline
Tweets			&	33774428\\
Users 			&	16099129\\
Hashtags 		&	1194691\\
Max tweets per user & 2380\\  
\hline
\end{tabular}
\caption{Characteristics of the dataset used in the experiments}
\label{tbl:dataset}
\end{table}

